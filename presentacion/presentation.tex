\documentclass[aspectratio=149]{beamer}
% The values that it accepts are: 1610, 169, 149, 54, 43 and 32, which stand for the ratios 16:10, 16:9, 14:9, 5:4, 4:3 and 3:2, respectively.

% Idioma y codificación en castellano
\usepackage[T1]{fontenc}
\usepackage[utf8]{inputenc}
\usepackage[english, spanish, es-tabla]{babel}

% Bibliografía
\usepackage{biblatex}
\addbibresource{bib/bibliografia.bib}

\usepackage{fontspec}
\usepackage{unicode-math}
\usepackage{lmodern}
\usepackage{setspace}
\usepackage{graphicx}
\usepackage{csquotes}
\graphicspath{{img/}}

%% Couleurs
\definecolor{comments}{rgb}{0.5,0.55,0.6} % commentaires
\definecolor{link}{rgb}{0,0.4,0.6}   % liens internes
\definecolor{url}{rgb}{0.6,0,0}      % liens externes
\definecolor{rouge}{rgb}{0.9,0,0.1}  % bandeau rouge UL
\definecolor{or}{rgb}{1,0.8,0}       % bandeau or UL

%% Hyperliens
\hypersetup{
    colorlinks = false,
    linktocpage = true,
    urlcolor = {url},
    linkcolor = {link},
    citecolor = {citation},
    pdfpagemode = {UseOutlines},
    pdfstartview = {Fit}}

%%% =========================
%%%  Nouveaux environnements
%%% =========================

%% Environnements pour les demo de code; tirés du document
%% principal. (L'environnement 'eqxample' ajoute des filets de part
%% et d'autre du bloc pour illustrer les marges.)
\newenvironment{demo}{%
    \begin{beamercolorbox}[wd=\linewidth,sep=6pt]{block body example}}
    {\end{beamercolorbox}}

\newenvironment{texample}[1][0.45\linewidth]{%
    \noindent\begin{minipage}{#1}%
    \def\producing{\end{minipage}\hfill\begin{minipage}{\dimexpr0.9\linewidth-#1}%
        \hbox\bgroup\kern-.2pt%
        \vbox\bgroup\parindent0pt\relax
        % The 3pt is to cancel the -\lineskip from \displ@y
        \abovedisplayskip3pt \abovedisplayshortskip\abovedisplayskip
        \belowdisplayskip0pt \belowdisplayshortskip\belowdisplayskip
        \noindent}
    }{%
        \par
        % Ensure that a lonely \[\] structure doesn't take up width less than
        % \hsize.
        \hrule height0pt width\hsize
        \egroup\kern-.2pt\egroup
        \end{minipage}%
        \par
    }

\newenvironment{eqxample}{%
    \noindent\begin{minipage}{.45\linewidth}%
    \def\producing{\end{minipage}\hfill\begin{minipage}{.45\linewidth}%
        \hbox\bgroup\kern-.2pt\vrule width.2pt%
        \vbox\bgroup\parindent0pt\relax
        % The 3pt is to cancel the -\lineskip from \displ@y
        \abovedisplayskip3pt \abovedisplayshortskip\abovedisplayskip
        \belowdisplayskip0pt \belowdisplayshortskip\belowdisplayskip
        \noindent}
    }{%
        \par
        % Ensure that a lonely \[\] structure doesn't take up width less than
        % \hsize.
        \hrule height0pt width\hsize
        \egroup\vrule width.2pt\kern-.2pt\egroup
        \end{minipage}%
        \par
  }

%% Simplfication de l'environnement 'quote' de beamer
\renewenvironment{quote}{%
    \begin{beamercolorbox}[wd=\linewidth,sep=6pt]{block body example}}
    {\end{beamercolorbox}}

%% Exercices
\newenvironment{exercice}{%
    \begin{frame}[fragile=singleslide]
    \frametitle{\faCogs\; Exercice}}{\end{frame}}
  
%%% =======
%%%  Varia
%%% =======

%% Longueurs pour la composition des pages couvertures avant et
%% arrière.
\newlength{\banderougewidth} \newlength{\banderougeheight}
\newlength{\bandeorwidth}    \newlength{\bandeorheight}
\newlength{\imageheight}
\newlength{\logoheight}

\mode<presentation> {
    \usetheme{ulaval}
    \setbeamercovered{transparent}
    \setbeamertemplate{navigation symbols}{}
}

\newenvironment{xframe}[2][]
    {\begin{frame}[fragile,environment=xframe,#1]
    \font
    \frametitle{#2}}
    {\end{frame}}

\AtBeginSection[]{
    \begin{frame}
    \Huge \centerline{\insertsection}
    \small \tableofcontents[currentsection, hideothersubsections]
    \end{frame}
}

\setcounter{tocdepth}{2}
\newcommand{\acc}[1]{\textcolor{ulred}{\textbf{#1}}}


\title[Automatización del proceso de adquisición de imágenes de redes sociales para sistemas de recomendación]{Automatización del proceso de adquisición de imágenes de redes sociales para sistemas de recomendación}

\author[Daniel González Alonso]{
    Autor: Daniel González Alonso\\[1ex] 
    Tutor: Ángel Manuel Guerrero Higueras}
\institute[Universidad de Valladolid]
{
    Máster Universitario en Inteligencia de Negocio \\[-4pt]
    y Big Data en Entornos Seguros
	%\medskip
	%{\emph{daniel.gonzalez.alonso@alumnos.uva.es}}
}



\titlegraphic{%
    \begin{center}
        \includegraphics[height=2cm]{img/escudoUBU} \hspace{1cm}
        \includegraphics[height=2cm]{img/escudoUVA} \hspace{1cm}
        \includegraphics[height=2cm]{img/escudoULE} \vspace{1cm}
    \end{center}}

\setbeamertemplate{title page}{%
    \begin{center}
        \vspace{0.4cm}
        {\textbf{\LARGE\inserttitle}}\\[0.8cm]
        \fontfamily{ptm}\selectfont
        {\insertauthor}\\[0.4cm]
        {\scriptsize\insertinstitute}\\[0.25cm]
        {\insertdate}\\[0.5cm]
        \inserttitlegraphic
    \end{center}}



\begin{document}

%==============================================================================
% Título
%==============================================================================
\begin{frame}[label=title, plain]
    \titlepage
\end{frame}

%==============================================================================
% TABLA DE CONTENIDOS
%==============================================================================
\begin{frame}[label=toc]{Tabla de contenidos}
 \setlength{\leftskip}{5cm}%
 \tableofcontents[subsectionstyle=hide]
\end{frame}

%==============================================================================
% INTRODUCCION
%==============================================================================
\section{Introducción}
\begin{frame}[label=intro]{Introducción}
    \begin{itemize}
        \item La monitorización y análisis de redes actualmente es un problema muy complejo debido a su tamaño y tráfico.
        \item Los métodos convencionales hacen uso de servidores centrales. Problemas:
        \begin{itemize}
            \item Limitaciones de cómputo
            \item Punto de fallo único.
        \end{itemize}
        \item Propuesta: Sistema de monitorización distribuido para el protocolo TCP empleando \textit{Spark Streaming}.
    \end{itemize}
\end{frame}

\begin{frame}[label=spark_streaming]{Spark Streaming}
    \begin{itemize}
        \item Hadoop es una plataforma que emplea un modelo MapReduce para procesar gran cantidad de datos.
        \item Para reducir el uso de la entrada/salida y con ello mejorar el rendimiento se creó Apache Spark.
        \item Ambas plataforma fueron ideadas para analizar grandes cantidades de datos en Batch mediante MapReduce.
        \item En la actualidad Spark ofrece la librería \textbf{Spark Streaming} para poder usarlo con streams de datos empleando una técnica conocida como \textit{micro-batching}.
    \end{itemize}
\end{frame}

%==============================================================================
% TRABAJOS RELACIONADOS
%==============================================================================
\section{Trabajos Relacionados}
\begin{frame}[label=relat]{Trabajos Relacionados}
    \begin{itemize}
        \item Sistema BRO para detectar intrusos en redes mediante un motor de eventos, el cual podía customizarse empleando un lenguage de scripting. Problema: un solo hilo.
        \item Gupta et al. propusieron un sistema para analizar redes en tiempo real mediante \textit{Spark Streaming}. Problema: solo funcionan mediante switchs programables.
        \item Karimi et al. propusieron un sistema distribuido de análisis de tráfico que captura paquetes y guarda sus cabeceras en un CSV que se procesa periódicamente. Problema: el rendimiento puede degradarse.
    \end{itemize}
\end{frame}

%==============================================================================
% ARQUITECTURA
%==============================================================================
\section{Arquitectura del Sistema}
\begin{frame}[label=arquitectura]{Arquitectura del Sistema de Monitorización}
    \begin{columns}
    \begin{column}{0.4\textwidth}
        \begin{enumerate}
            \item Collector
            \item Sistema de Mensajería
            \item Procesador de streams
        \end{enumerate}
    \end{column}
    \begin{column}{0.7\textwidth}
        \begin{figure}
            \centering
            \includegraphics[width=1.0\textwidth]{img/arquitectura.png}
            \caption{Arquitectura del Sistema de Monitorización}
            \label{fig:arch}
        \end{figure}
    \end{column}
    \end{columns}
\end{frame}

%==============================================================================
% MONITORIZACION DE MENSAJES TCP
%==============================================================================
\section{Monitorización de Mensajes TCP}

\subsection{Métricas}
\begin{frame}[label=metricas]{Métricas}
    \begin{itemize}
        \item Throughput: número y longitud de los segmentos TCP por segundo.
        \item Ratio de retransmisón y out-of-order: Indica el ratio de mensajes que han llegado desordenados o que han sido retransmitidos. Si este ratio es malo, es un síntoma de mal rendimiento en la red.
        \item RTT: es el tiempo que tarda la comunicación en viajar desde el origen al destino y su correspondiente respuesta ACK.
    \end{itemize}
\end{frame}

\subsection{Sistema Propuesto}
\begin{frame}[label=sistema]{Sistema Propuesto}
    \begin{enumerate}
        \item Preprocesamiento: Los Collectors capturan segmentos TCP de la red y almacenan de cada uno una tupla, generando el stream \texttt{DStream0}. Este después es preprocesado para generar nuevas tuplas que añaden el siguiente número de secuencia esperado, creando otro stream \texttt{DStream1}.
        \item Cálculo de Throughput: de cada tupla de \texttt{DStream1} se calcula el número total de paquetes obtenidos entre un origen y un destino y la longitud total de los paquetes mediante \texttt{reduceByKey}. Esta información es almacenada en un nuevo stream \texttt{DStream2}.
    \end{enumerate}
\end{frame}
\begin{frame}{Sistema Propuesto}
    \begin{enumerate}
        \setcounter{enumi}{2}
        \item Retransmisión y out-of-order: A partir de \texttt{DStream1} se calcula el número de retransmisiones y de elementos desordenados mediante el siguiente algoritmo. El resultado se guarda en el stream \texttt{DStream3}.
        \begin{columns}
        \begin{column}{0.4\textwidth}
        	\includegraphics[width=1.0\textwidth]{img/alg1_1.png}
        \end{column}
        \begin{column}{0.4\textwidth}
        	\includegraphics[width=1.0\textwidth]{img/alg1_2.png}
        \end{column}
        \end{columns}
    \end{enumerate}
\end{frame}
\begin{frame}{Sistema Propuesto}
    \begin{columns}
    \begin{column}{0.6\textwidth}
        \begin{enumerate}
            \setcounter{enumi}{3}
            \item RTT: los segmentos TCP y los segmentos ACK se agrupan mediante \texttt{groupByKey}, lo que permite calcular el RTT mediante el siguiente algoritmo. El resultado obtenido se almacena en el stream \texttt{DStream4}.
            \item Etapa final: se unen los datos obtenidos en el \texttt{DStream2}, \texttt{DStream3} y \texttt{DStream4} mediante una operación de \texttt{join} para poder mostrar los resultados al usuario.
        \end{enumerate}
    \end{column}
    \begin{column}{0.5\textwidth}
        \begin{figure}
            \centering
            \includegraphics[width=1.0\textwidth]{img/alg2.png}
            %\caption{Algoritmo RTT}
            \label{fig:alg2}
        \end{figure}
    \end{column}
    \end{columns}
\end{frame}

%==============================================================================
% RESULTADOS
%==============================================================================
\section{Resultados}
\begin{frame}[label=result]{Resultados}
	\begin{figure}
	    \centering
	    \includegraphics[width=0.7\textwidth]{img/results.png}
	    \caption{Resultados}
	    \label{fig:results}
	\end{figure}
\end{frame}

%==============================================================================
% CONCLUSIONES
%==============================================================================
\section{Conclusiones}
\begin{frame}[label=conclu]{Conclusiones}
    \begin{itemize}
        \item Los métodos convencionales que empleaban un único servidor para analizar el estados de las redes ya no son viables.
        \item Los métodos actuales que emplean plataformas Big Data están más centrados en el procesamiento de los datos de forma offline.
        \item En este estudio\cite{8268735} se ha propuesto y probado un sistema de monitorización de redes mediante el protocolo TCP empleando la plataforma Spark Streaming que ofrece un buen rendimiento y robustez.
    \end{itemize}
\end{frame}

%==============================================================================
% Bibliografía
%==============================================================================
\begin{frame}{Bibliografía}
    \printbibliography
\end{frame}

\end{document}