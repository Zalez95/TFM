\capitulo{6}{Trabajos relacionados}

En este apartado se va a exponer el estado del arte, o los trabajos encontrados relacionados con el tema del presente proyecto con el fin de conocer si ya existe algún trabajo anterior que intente resolver el mismo problema que el aquí presentado.

\section{Metodología de trabajo}

Durante este proceso de documentación, lo primero que se llevó a cabo fue una búsqueda preliminar con el fin de conocer si existía algún artículo previo donde se tratara en específico el análisis de emociones en imágenes de Instagram para recomendar posts o lugares a los que visitar. Para ello se hizo uso de bases de datos y exploradores gratuitos como Google Scholar, IEEE Xplore y ResearchGate. El resultado de esta búsqueda fue negativo, pero sí que se encontraron multitud de trabajos en los que se intentaba llevar a cabo análisis de sentimiento de comentarios de diversas redes sociales para saber la reacción de los usuarios que hacían turismo en cierto lugares en específico como \cite{8720960}, o trabajos generales sobre análisis de sentimiento para ayudar a recomendadores como \cite{techniques_media_based_recom} o \cite{recom_sys_sen_analysis}.

Como este trabajo preliminar no fue satisfactorio, se decidió ampliar los criterios de búsqueda, llevando a cabo un proceso de \textit{map-review} con el objetivo de recopilar los trabajos previos más relevantes para el presente proyecto. Debido a que la cantidad de artículos relacionados con análisis de sentimiento es abrumadora, para llevar a cabo el map-review se decidió emplear la metodología PRISMA (\textit{Preferred Reporting Items for Systematic Reviews and Meta-Analyses}) inspirado por el artículo \cite{recom_metodo_tutor}. Esta metodología es empleada para llevar a cabo revisiones sistemáticas de conjuntos de artículos, permitiendo descartar aquellos que no son relevantes sin tener que leerlos todos por completo, agilizando todo el proceso. En resumen, esta metodología cuenta de varias etapas:

\begin{enumerate}
    \item Planificación de búsqueda: Hay que comprobar que no haya ya un trabajo previo que resuelva la misma pregunta. En nuestro caso como se comentó al inicio del apartado, aparentemente no parece haber un trabajo anterior similar.
    \item Proceso de búsqueda: En esta fase hay que generar las cadenas de búsqueda que se emplearan en los buscadores de artículos. En nuestro caso se decidió emplear las siguientes cadenas:
    \begin{itemize}
        \item \texttt{(Sentiment Analysis OR Emotion Recognition) AND Social Media AND Tourism}
        \item \texttt{(Sentimen Analysis OR Emotion Recognition) AND Images \\
        AND Recommendation System}
    \end{itemize}
    
    Además de las anteriores cadenas de búsqueda, se ha decidido también aplicar otros filtros a mayores como:
    
    \begin{itemize}
        \item El acceso al artículo se ha de poder hacer de forma gratuita.
        \item El artículo ha de ser ``actual'', de entre 2014 y 2022.
    \end{itemize}
    
    De este modo se obtuvieron un total de 22 artículos a analizar en las siguientes fases.

    \item Selección de muestras. En esta fase se parte de los artículos anteriormente obtenidos y, tras descartar aquellos que estén duplicados, se decide si descartar alguno tras leer el \textit{abstract} basándose en una serie de criterios de inclusión y de exclusión. En este caso se ha decidido que:

    \begin{itemize}
        \item Se incluye si:
        \begin{itemize}
            \item El abstract está relacionado con análisis de sentimientos en redes sociales
            \item El abstract está relacionado con análisis de sentimientos en imágenes
            \item El abstract está relacionado con sistemas de recomendación de turismo
        \end{itemize}
        
        \item Se excluye si :
        \begin{itemize}
            \item El idioma del artículo no es inglés ni español.
            \item El artículo no se puede acceder gratuitamente
            \item El artículo es una recopilación de trabajos previos
            \item El artículo no está relacionado con análisis de sentimientos
            \item La fuente de datos del artículo no son redes sociales
        \end{itemize}
    \end{itemize}

    Si tras leer el abstract es detectado algún criterio de exclusión o no cumple con ningún criterio de inclusión, el artículo es directamente es descartado. Después de descartar por el abstract, los artículos restantes son leídos por completo y se decide si incluirlos o no basándose en un test con las siguientes preguntas binarias (Sí o No):

    \begin{enumerate}
        \item ¿Emplea herramientas de análisis de sentimientos?
        \item ¿Emplea análisis de sentimiento basado en imágenes?
        \item ¿Emplea análisis de sentimiento en redes sociales?
        \item ¿Hace referencia a sistemas de recomendación de turismo?
        \item ¿Contiene tablas y gráficos de resultados?
    \end{enumerate}
    
    Solo son aceptados aquellos artículos que superen el anterior test con 3 sí-es. Tras este proceso de cribado se llegó a la conclusión de que solo 6 de los 22 trabajos considerados anteriormente eran relevantes para el nuestro.
\end{enumerate}

De los trabajos finalmente considerados, estos podrían ser divididos en tres categorías: por un lado tenemos los trabajos donde se busca recomendar contenido, posts o usuarios que seguir, por otro lado, tenemos trabajos donde se recomiendan lugares que visitar, y por último tenemos trabajos donde se busca analizar las opiniones sobre un destino turístico.

\section{Recomendación de contenido}

Dentro del primer grupo se encuentran trabajos como \cite{sent_analysis_facebook_user_recom} donde se investiga el uso de análisis de sentimiento en la red social Facebook para mejorar los resultados de las recomendaciones de usuarios/amigos. Para llevar a cabo este objetivo, en este trabajo se analiza el texto de las publicaciones de los usuarios para extraer conceptos de las mismas, la subjetividad u objetividad de las frases, y los sentimientos encontrados en los documentos. A la hora de recomendar o no a un usuario, en este artículo se propone una novedosa función de peso que tiene en cuenta el sentimiento de un usuario hacia un concepto así como el número de publicaciones de tal usuario acerca del mismo. La evaluación del método propuesto se hizo empleando un conjunto de datos de más de 6 millones de publicaciones extraídas de Facebook, dando un resultado superior a otros métodos ya establecidos.

Por otro lado, en el artículo \cite{pers_tweet_recomendation} se propone un sistema de recomendación de \textit{tweets} personalizado dentro del dominio de la salud. Para ello hace uso del análisis de sentimiento para la creación de un perfil de usuario basándose en el historial en la red social del usuario. El sistema propuesto consta de dos módulos, el primero se encarga de generar un perfil de usuario mediante la extracción información de su cuenta, sus intereses en temas de salud y los patrones de emoción a lo largo del tiempo, mientras que el segundo recoge datos públicos de Twitter, lleva a cabo un procesamiento de lenguaje natural y los clasifica para poder recomendarlos a los usuarios basándose en su correspondiente perfil. Este sistema se probó contra casi 1 millón de \textit{tweets} de distintas categorías, alcanzando una precisión del 96\%.

\section{Recomendación de lugares}

Dentro del segundo grupo existen trabajos como \cite{8796367} donde se presenta un sistema de recomendación de turismo personalizado teniendo en cuenta las preferencias del usuario. Su método denominado SMTM divide el problema en dos dominios, por un lado la extracción de los tópicos relacionados con una atracción turística, y por otro las preferencias del turista respecto a las atracciones a partir de textos e imágenes. Por último trata de proyectar los resultados del dominio del turista en el dominio de la atracción. Para probar el modelo generado con otros modelos se emplearon dos datasets, el primero generado a partir de de información del portal web TripAdvisor, y el segundo a partir de datos de la web Trip, en ambos casos los resultados obtenidos fueron mejores que el resto de métodos comparados.

Además, en el artículo \cite{recom_mech_under_emph} se propone un sistema de recomendación de turismo con dos objetivos, primero satisfacer las expectativas del turista, y segundo ayudar a las agencias de marketing a orientar sus promociones. Este sistema de recomendación se planteó con el objetivo de poder recomendar tanto lugares altamente valorados como aquellos que son pasados por alto por muchos turistas pero que merecen la pena, a los que llama lugares poco enfatizados, y cabe destacar que este sistema de recomendación es personalizado, teniendo en cuenta el historial del usuario. El sistema consta de varios módulos, el primero se usa para llevar a cabo una extracción de tópicos de las reviews publicadas en Google y en TripAdvisor. El segundo lleva a cabo un análisis de sentimiento mediante máquinas de vectores soporte. Por último emplea una red neuronal artificial para recomendar los lugares a los usuarios, empleando una función de optimización para conseguir el objetivo de recomendar lugares poco enfatizados. Para probar el sistema se empleó un dataset con casi 150.000 publicaciones, llegando a obtener tener una precisión del 94\%.

\section{Análisis sobre destinos turísticos}

A diferencia de los trabajos anteriores donde el objetivo era el estudio o creación de sistemas de recomendación, existen trabajos como \cite{tourism_dest_rec_geolocation} donde se compara el rendimiento de múltiples métodos actuales de aprendizaje profundo con el fin de llevar a cabo un análisis de sentimiento de los tweets relacionados con Cilento, un popular destino turístico del sur de Italia. En este artículo se llegan a comparar 4 tipos de redes neuronales profundas empleando un dataset de casi 20000 tweets en inglés e italiano, cuyo sentimiento tuvo que ser clasificado manualmente de forma previa. Además de la comparación de estos métodos, en dicho artículo consiguen generar un mapa mediante la geolocalización de los tweets para poder así visualizar el sentimiento promedio en ciertas regiones del municipio.

También en \cite{su13116015} se lleva a cabo un análisis sobre el turismo en España y la percepción que tienen los turistas chinos empleando publicaciones en portales y redes sociales de China con el objetivo de medir la calidad de los distintos destinos turísticos. Para ello obtuvieron una dataset de casi 40 mil publicaciones, y llevaron a cabo una extracción de tópicos relacionados con las publicaciones así como análisis de sentimientos mediante procesamiento de lenguaje natural.