\capitulo{1}{Introducción}

El turismo es una de las actividades económicas más importantes de España, en 2019 suponía el 12,4\% del PIB del país, siendo el tercer país más visitado del mundo. Con la pandemia de COVID-19 el sector turístico se vio gravemente afectado por las restricciones de movilidad y el propio temor a la enfermedad. En la actualidad este sector todavía no ha recuperado los niveles previos a la pandemia. Si en 2019 se recibieron 83,5 millones de turistas, en 2021 esa cifra ha sido aproximadamente de 31,2 millones \cite{turismo_espana}.

En el caso de Valladolid en concreto, esta ciudad no partía de una situación ventajosa en comparación con otras ciudades de España, ubicándose en la zona media-baja de la tabla. Y al igual que el resto del país, la pandemia ha reducido el número de visitantes: la provincia has pasado de unos 51000 en 2019 a aproximadamente 33000 en 2021 de acuerdo al INE. 

Para recuperar los niveles de turismo previos (y seguir aumentándolos) es necesario potenciar el turismo, y uno de los posibles medios que se pueden aprovechar es internet, y en concreto las redes sociales. Saber explotar la información de las redes sociales puede tener una influencia enorme a la hora de mejorar el número de visitantes que pueda tener una ciudad. Pero abordar esta tarea es una asunto complicado.

Con la aparición y popularización de las redes sociales, cada vez más gente comparte los lugares que visita y sus opiniones. Las publicaciones que hace la gente en estas plataformas ayudan a otras personas a descubrir nuevas experiencias y lugares que visitar. También permite a las personas conocer más en detalle aquellos sitios a los que planea viajar. Por otro lado, todos estos contenidos generados por los usuarios son una fuente de información incalculable para todas aquellas empresas que están buscando mejorar sus productos vacacionales.

Una de las redes sociales más usadas en la actualidad es Instagram. Esta red social permite compartir publicaciones principalmente en forma de imagen o vídeo. La forma en que se muestra el contenido en esta red social la hace ideal para compartir lugares visitados o experiencias vividas, y aunque no sea el único tipo de publicaciones que se comparten en esta red social, si que es bastante común encontrarlo. Es por ello que los efectos que puede tener esta red social a la hora de difundir el potencial turístico que tiene una ubicación son muy importantes.

Cuando se llevan a cabo búsquedas en Instagram, las publicaciones que se muestran se basan en tres criterios, la cadena de búsqueda, los cuentas y hashtags que se siguen o visita y la popularidad de las publicaciones encontradas. Un problema que tiene esta forma de mostrar las publicaciones es que si se pretende hacer turismo a un lugar lejano, es posible que ninguno de tus contactos tenga influencia en los resultados, y es más que probable que Instagram tan solo muestre las publicaciones más populares del destino que estás buscando.

Es por ello que en este trabajo se pretende investigar el posible uso de reconocimientos de imágenes y análisis de sentimiento en las publicaciones de Instagram para facilitar el descubrimiento de actividades y lugares recomendables en Valladolid.

\section{Organización de la memoria}
La memoria se encuentra dividida en los siguientes capítulos:

\begin{enumerate}
    \item Introducción: se presenta el proyecto y la relación del turismo con las redes sociales.
    \item Objetivos del proyecto: se describen los distintos objetivos que se buscan cumplir con el desarrollo del proyecto.
    \item Conceptos teóricos: se introducen los conceptos teóricos necesarios para compresión de la memoria.
    \item Técnicas y herramientas: se enumeran y describen las múltiples herramientas, servicios, librerías y APIs empleadas para el desarrollo del proyecto.
    \item Aspectos relevantes del desarrollo del proyecto: se expone el trabajo desarrollado, los problemas encontrados y sus soluciones.
    \item Trabajos relacionados: se comenta el estado del arte del proyecto.
    \item Conclusiones y Líneas de trabajo futuras: se indican los resultados, las conclusiones y las posibles mejoras del proyecto desarrollado.
\end{enumerate}

\section{Organización de los apéndices}
