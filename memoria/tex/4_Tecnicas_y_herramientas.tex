\capitulo{4}{Técnicas y herramientas}

Este apartado tiene el objetivo de enumerar y describir las múltiples herramientas utilizadas para el desarrollo del presente proyecto.

Durante la fase inicial se trató de llevar a cabo una prueba de concepto empleando el lenguaje de programación Python para comprobar la viabilidad de la idea y la dificultad de implementación de descarga de datos y el posterior proceso de análisis de sentimientos. La decisión de usar este lenguaje de programación se basa en que es un lenguaje que conozco ya que he trabajado con éste en el pasado, y que considero sencillo y especialmente útil para llevar a cabo prototipos de forma rápida, ya que al ser muy popular actualmente, es muy fácil encontrar librerías y recursos en la web en caso de duda.

\section{Instalooter}

El primer paso que se llevo a cabo para esta prueba de concepto fue investigar las posibles herramientas para llevar a cabo la descarga de datos de la página web de Instagram. Tras hacer varias pruebas con múltiples scrapers web encontrados en GitHub se decidió emplear Instalooter, el cual dispone tanto de una interfaz de programación para Python como de un cliente de línea de comandos. El resto de scrapers encontrados no parecían funcionar, posiblemente porque la web de Instagram haya sido actualizada recientemente y éstos programas no hayan sido actualizados de acorde a estos cambios en el momento en que hice la prueba. La API de Instalooter pone a disposición del usuario de varios \textit{looters} ya sea para descargar las publicaciones llevadas a cabo por un usuario en su perfil o para descargar publicaciones relacionadas con un Hashtag, y permite descargar tanto las imágenes como la descripción y metadatos de las mismas en formato JSON. Además, antes de descargar la información de una publicación Instalooter comprueba que no haya sido ya descargada, con lo que puede ejecutarse repetidamente sobre un perfil o un Hashtag sin peligro de obtener resultados duplicados.

\section{Servicios en la Nube}

Respecto al almacenamiento y procesado de los datos descargados de las publicaciones de Instagram, se trató desde un principio de usar computación en la nube debido a la facilidad de uso y de escalado de la capacidad de almacenamiento y computación en caso de que fuese necesario.

Inicialmente por recomendación del tutor se trató de emplear los servicios de Google Cloud. Esta plataforma de Google Cloud provee diversos servicios junto con sus respectivas interfaces de programación compatibles con múltiples lenguajes de programación. Entre los servicios relevantes para el presente proyecto cabe destacar el almacenamiento en la nube con Cloud Storage, servicios de procesamiento de datos y creación de máquinas virtuales con Compute Engine, servicios de visión artificial mediante Cloud Vision, almacenamiento de bases de datos en Cloud SQL, etc. Todos estos servicios se pueden probar de forma gratuita con tan solo crearse una cuenta en la plataforma. Esto es debido a que Google ofrece 300 USD para invertir libremente a toda cuenta recién creada durante un periodo de 3 meses, 400 USD si se demuestra que eres alumno \cite{google_cloud}. Este crédito inicial se va gastando según se vaya haciendo uso de los servicios, y una vez consumido o pasado el periodo de 3 meses, Google empieza a cobrar por el uso de sus servicios. Debido al alcance de este proyecto y los precios que tienen los servicios de Google, el límite de dinero no parece preocupante, aunque el de tiempo si que puede llegar a serlo.

Como se ha comentado con anterioridad, para emplear los servicios de Google Cloud, Google pone a disposición de los usuarios APIs de forma gratuita para distintos lenguajes de programación, entre los cuales se encuentra Python. Mediante este lenguaje de scripting y la API de Cloud Vision \cite{api_google_vision}, se llevaron a cabo unas pruebas iniciales donde se pudo comprobar que la información que se puede obtener mediante el uso de la API de Google Cloud Vision si que incluye cierta capacidad de análisis de emociones, permitiendo obtener etiquetas de emociones como alegría, tristeza, enfado, sorpresa, etc. junto con su \textit{likelihood}, pero no incluye información como edad y género de las personas, capacidad que fue eliminada de este servicio de forma bastante reciente  \cite{archive_google_gender}. Esto es un problema ya que se pretendía hacer uso de este tipo de información a la hora de recomendar publicaciones o lugares al usuario. Debido a esto se decidió comparar las alternativas existentes a estos servicios en la nube de Google.

A continuación se hace un pequeño resumen de los servicios ofrecidos por las alternativas valoradas y que posiblemente se puedan llegar a necesitar en el proyecto. Cabe destacar que se seleccionaron solo estas alternativas en base a los servicios necesarios y al requerimiento de que la plataforma usada fuera gratuita u ofreciese un periodo de prueba, ya sea para estudiantes o no, lo suficientemente extenso como para poder desarrollar el proyecto sin complicaciones.

\begin{landscape}
\begin{table}[]
    \centering
    \resizebox{1.02\columnwidth}{0.55\textwidth}{%
        \begin{tabular}{|p{0.12\textwidth}|p{0.2\textwidth}|p{0.15\textwidth}|p{0.15\textwidth}|p{0.15\textwidth}|p{0.21\textwidth}|p{0.21\textwidth}|p{0.21\textwidth}|p{0.21\textwidth}|}
        \hline
        \multirow{2}{=}{Servicios} & \multirow{2}{=}{Límite de uso}	& \multicolumn{3}{|c|}{API de Visión Artificial}	& \multirow{2}{=}{Almacenamiento en bucket/blobs gratuito}	& \multirow{2}{=}{Almacenamiento en Base de Datos SQL}	& \multirow{2}{=}{Almacenamiento en Base de Datos NoSQL}	& \multirow{2}{=}{Máquinas Virtuales}	\\ \cline{3-5}
        	&	& Límite de uso	& Análisis de Sentimiento	& Reconoci\-miento de edad y género &	&	&	&	\\ \hline
        Google Cloud	& 300 USD durante 3 meses	& 1000 unidades al mes	& Si, con tags alegría, tristeza, enojo, sorpresa	& No	& Si, con 5 Gbs al mes en Google Cloud Storage	& Si, con motor de base de datos MySQL, PostgreSQL y SQL Server	& Si, bases de datos documentales como Cloud Datastore, Cloud Firestore... y clave/valor con MongoDB, Bigtable...	& Si, Máquina virtual en Compute Engine empleando el saldo inicial	\\ \hline
        Microsoft Azure	& 12 meses para servicios gratuitos + 200 USD para servicios no gratuitos durante 1 mes & Hasta 30.000 transacciones al mes & Si, con tags felicidad, tristeza, neutralidad, ira, desprecio, asco, sorpresa y temor	& Si	& Si, con 5 Gbs en Azure Blob Storage y 5 GBs en Azure File Storage 5 & Si, con Azure Database for MySQL (750 horas al mes), Azure Database for PostgreSQL (750 horas al mes) y SQL Database (250 Gbs)	& Si, Azure Cosmos DB (25 Gbs)	& Si, Máquina virtual B1S (1 núcleo, 1 GB RAM y 4GB de almacenamiento) con Linux 750 horas al mes	\\ \hline
        Amazon Web Services (AWS)	& 12 meses para la capa gratuita	& 5000 imágenes al mes	& Si, con tags feliz, triste, enfado, confuso, disgustado, sorprendido, calma, miedo, desconocido & Si	& Si, con 5 Gbs al mes en Amazon S3	& Si, se puede usar RDS (750 horas al mes) con motor de base de datos MySQL, MariaDB, Oracle, PostgreSQL, SQL Server y Amazon Aurora & Si, bases de datos documentales con DocumentDB (compatible con MongoDB, solo 1 mes de prueba) y clave/valor con DynamoDB (25 GB gratuitos) & Si, Máquina virtual en EC2 (1 CPU, 1GB de RAM, 1 CPU y 1GB de RAM, 30 GBs SSD) con Linux 750 horas al mes \\ \hline
        \end{tabular}%
    }
    \caption{Servicios en la Nube: Alternativas}
    \label{tab:cloud_services}
\end{table}
\end{landscape}

Hay que destacar que se intentó llevar a cabo pruebas en las distintas plataformas mediante sus respectivas interfaces de programación. Durante las pruebas, con los servicios de Microsoft Azure no se consiguió llegar a crear una máquina virtual de tipo \texttt{B1S} gratuita, siempre salían como no disponibles. Tras revisar la página de preguntas y respuestas de Microsoft parece que a mucha gente le ha sucedido lo mismo, el problema parece ser la alta demanda de este tipo de máquinas. Debido a este problema y la imposibilidad de obtener la edad y el genero en los servicios de Google Vision, se decidió que lo mejor era emplear los servicios de \textbf{Amazon Web Services}.

\subsection{Amazon S3}
Amazon S3, \textit{Simple Storage Service} por sus siglas en inglés, es un servicio de almacenamiento de objetos en la nube de Amazon. El almacenamiento en S3 permite almacenar cualquier tipo de objeto de forma escalable, con baja latencia y alta disponibilidad. El acceso a este servicio puede llevarse a cabo mediante la consola web de Amazon AWS, el SDK de AWS o mediante APIs REST.

Cada objeto almacenado en S3 se ubica dentro de un recurso denominado ``bucket'' y puede llegar a ocupar hasta 5TBs \cite{amazon_s3}. Estos objetos se organizan mediante prefijos y se les puede añadir hasta 10 etiquetas clave-valor. Además, se puede habilitar que la modificación de los objetos almacenados en S3 pueda ser supervisada mediante sistemas de control de versiones así como habilitar Multi-Factor Athentication para evitar el borrado accidental. Cabe destacar también que los objetos pueden ser replicados en varias regiones para reducir la latencia de acceso si nuestra aplicacion así lo requiere.

Amazon S3 tiene varias clases de almacenaniento. Dependiendo del caso de uso y la forma en que se va a acceder a los datos puede ser más conveniente usar una clase u otra con el fin de reducir la latencia y el coste de servicio de nuestra aplicación. Estas clases son las siguientes:
\begin{itemize}
    \item \textit{S3 Intelligent-Tiering}: preferible para aplicaciones con patrones de acceso desconocidos o cambiantes ya que optimiza de forma automática los costes de almacenamiento.
    \item \textit{S3 Standard}: preferible para los datos de acceso frecuente.
    \item \textit{S3 Standard-Infrequent Access} y \textit{S3 One Zone-Infrequent Access}: preferibles para los datos de acceso poco frecuente.
    \item \textit{S3 Glacier Instant Retrieval}, \textit{S3 Glacier Flexible Retrieval} y \textit{S3 Glacier Deep Archive}: preferibles para el archivado de datos, siendo preferible uno u otro en base a las necesidades de tiempo de recuperación (\textit{instant} en milisegundos, \textit{flexible} en minutos, \textit{deep archive} en horas).
    \item \textit{S3 Outposts}: útil para aquellos casos donde existe alguna normativa que nos obliga a mantener los datos fuera de alguna región en concreto.
\end{itemize}

Además Amazon S3 permite regular el acceso a los datos mediante listas de control de acceso (ACL), políticas de bucket, etc. Cabe destacar que el servicio IAM de AWS simplifica mucho la administración del acceso a los datos.

\subsection{Amazon EC2}

Amazon EC2, Elastic Compute Cloud por sus siglas en inglés, es un servicio de AWS que permite crear instancias o máquinas virtuales en la nube con alta disponibilidad y a un coste asequible. EC2 permite elegir por un lado el hardware que más se adapte a nuestras necesidades, como arquitectura de CPU, modelo de CPU, memoria RAM, capacidad y tipo de almacenamiento, ancho de banda, etc. Y por otro el Software, con disponibilidad de distintos sistemas operativos como Microsoft Windows, Amazon Linux, Red Hat, Debian. Ubuntu, SUSE, etc.

Una de las ventajas de usar EC2 es su flexibilidad que permite reducir costes adaptándose a las necesidades de cómputo del usuario, teniendo planes que permiten ejecutar el cómputo solo a ciertas horas, planes que permiten ajustarse a la demanda del usuario, planes para cierta cantidad de uso por hora constante, etc. Además, la capa gratuita de AWS ofrece hasta 750 horas de cómputo al mes durante el primer año.

Cabe destacar que EC2 ofrece distintos tipos de instancias \cite{ec2_instances}:
\begin{itemize}
    \item Uso general: es la opción más equilibrada.
    \item Optimizadas para informática: con procesadores de alto rendimiento.
    \item Optimizadas para memoria: para aplicaciones que necesitan trabajar con muchos datos en memoria.
    \item Informática acelerada: con aceleradores hardware como GPUs.
    \item Optimizadas para almacenamiento: para aplicaciones que dependen mucho de la entrada/salida.
\end{itemize}

\subsection{DynamoDB}

DynamoDB es una base de datos NoSQL de alta disponibilidad de Amazon. Esta tecnología surgió debido a las necesidades particulares de Amazon, ya que esta empresa tiene cientos de servicios descentralizados con picos de demanda muy elevados. Para dar soporte a estas necesidades de escalabilidad y de alta disponibilidad Amazon ya había desarrollado con anterioridad otros sistemas de almacenamiento como S3, pero como existen muchos servicios de Amazon cuyos patrones de acceso son únicamente clave-valor, así que decidieron crear un sistema de almacenamiento en específico para este tipo de casos que tuviera menor latencia. Otros requisitos que tenía que cumplir este sistema era que tenía que poder ejecutarse sobre hardware básico (barato y extendido), e intentar cumplir con las propiedades ACID: Atomicidad, Consistencia, Aislamiento y Durabilidad. Como cumplir con todas estas propiedades suelen provocar que la disponibilidad no sea muy elevada, DynamoDB sacrifica ligeramente la propiedad de Consistencia.

Para la distribución del conjunto de datos en distintos nodos, DynamoDB emplea una clave de particionamiento, que es implementada como una clave hash. Por otro lado, para garantizar la durabilidad de los datos esta herramienta los replica en varios nodos, y cualquier cambio en los datos se propaga de forma asíncrona ya que DynamoDB garantiza la consistencia de los datos de forma eventual.

Para usar DynamoDB en AWS se dispone de Amazon DynamoDB, que es un servicio que nos permite utilizar esta base de datos NoSQL sin la necesidad de tener que gestionar y mantener nuestro propio servidor, ya que se ejecuta y almacena en la nube de Amazon. El coste de este servicio es bajo demanda, pero la capa gratuita de AWS nos permite utilizar hasta 25GBs de forma permantente.

\subsection{AWS Lambda}

AWS Lambda es un servicio de Amazon que permite ejecutar código ``sin servidor'', es decir, sin la necesidad de tener que gestionar una instancia o un clúster \cite{aws_lambda}. Lambda permite ejecutar código de forma nativa de múltiples lenguajes de programación, como puede ser Java, Node.js, Go, Ruby, Python, etc. Este servicio ejecuta nuestro código como respuesta a eventos, como por ejemplo la modificación de un fichero de S3, cambios en nuestra base de datos de DynamoDB, solicitudes HTTP, eventos temporales, etc. 

Como se ha comentado, la principal característica y ventaja de Lambda es no tener que gestionar nuestro propio servidor, teniendo que llevar a cabo sus actualizaciones y demás inconvenientes. Pero además ofrece otras ventajas como puede ser la completa integración con otros servicios de AWS, el escalado automático, la alta disponibilidad, un tiempo de despliegue e inicio muy corto y un costo de operación reducido puesto que el modelo de Lambda es de pago por uso. Además, Amazon ofrece 1 millón de llamadas gratuitas al mes.

\subsection{Amazon API Gateway}

Amazon API Gateway es un servicio de Amazon que permite la publicación de APIs de forma sencilla \cite{aws_apigateway}. Este servicio facilita la creación, monitorización y mantenimiento de las APIs, lo que se traslada en una reducción del coste de horas que se tienen que dedicar a este tipo de tareas. Además este servicio ofrece alta disponibilidad y baja latencia. Las APIs que se pueden publicar en API Gateway pueden ser WebSocket o RESTful.

Este servicio se puede integrar con otros servicios de AWS como Lambda para llamar a nuestros métodos fácilmente. Se trata de un servicio de pago por uso, pero Amazon ofrece 1 millón de llamadas a las APIs al mes de forma gratuita durante el primer año.

\section{Grafana}

Grafana es una herramienta multiplataforma de código abierto para la visualización y análisis de datos. Para ello Grafana permite crear, explorar y compartir cuadros de mando \cite{wiki_grafana}\cite{github_grafana}. Entre las múltiples características que ofrece esta herramienta cabe destacar:

\begin{itemize}
    \item Es una herramienta flexible que permite crear diversos tipos de gráficos con multitud de opciones, como histogramas, mapas geográficos, gráficos de tarta, de barras, etc.
    \item Permite la creación de dashboard dinámicos que pueden ser reusados fácilmente. Los cuadros de mando se pueden exportar en formato JSON de forma que se puede copiar y pegar de una forma sencilla.
    \item Se pueden añadir variables que pueden ser usadas como filtros.
    \item Permite la creación de consultas de forma dinámica y comparar diferentes rangos de tiempo.
    \item Puede ser usado para explorar logs en vivo.
    \item Permite definir alertas en base a distintas métricas.
    \item Es capaz de mezclar datos de distintos orígenes en un mismo dashboard.
\end{itemize}

Grafana es extensible mediante plugins que permiten añadir funcionalidad así como conectores con distintos tipos de fuentes de datos. Al ser código abierto existe una gran comunidad detrás de Grafana que ayuda tanto al desarrollo y soporte de la herramienta en sí, como para la creación de nuevos plugins. En la actualidad hay conectores para múltiples fuentes de datos como pueden ser hojas de cálculo, bases de datos MySQL, Oracle, MongoDB, Splunk...

\section{Otras herramientas}

Además de las anteriores herramientas utilizadas para la descarga, almacenamiento y procesamiento de datos, se han utilizados otras herramientas y utilidades para llevar a cabo tanto la implementación de los programas como la realización de la memoria, entre las que cabe destacar:

\begin{itemize}
    \item virtualenv: herramienta empleada en Python para crear entornos aislados, evitando que las actualizaciones e instalaciones llevadas a cabo para otros proyectos afecten al actual.
    \item Visual Studio Code: editor de código fuente usado para la implementación de los scripts.
    \item Git: Sistema de Control de Versiones empleado tanto con el código como con la memoria del proyecto. El repositorio se decidió almacenar en la forja GitHub para poder compartirlo fácilmente con el tutor, y además para poder acceder a sus herramientas de seguimiento del proyecto.
    \item \LaTeX: Sistema de composición de textos de alta calidad empleado para la creación de la memoria. LaTeX se usó junto con la web Overleaf, ya que ésta evita tener que instalar todo el entorno de compilación de LaTeX junto con sus dependencias, y además permite sincronizar los cambios con GitHub fácilmente.
\end{itemize}
