\capitulo{3}{Conceptos teóricos}


Computer Vision\\
Análisis de Emociones\\
Sistemas de Recomendación\\
Crawler\\
Instagram\\
Dashboard/cuadro de mandos\\









\subsection{Emociones y Computación afectiva}

Las emociones son un conjunto de reacciones neurofisiológicas que sufre un individuo al ser expuesto a algún objeto, lugar, persona, suceso o idea. Las emociones son complejas, involucran múltiples componentes como el experiencia previa de los individuos, características subjetivas, procesos cognitivos, etc. En múltiples ocasiones las emociones son el origen de los cambios en la motivación y el comportamiento de los individuos. A lo largo de los últimos años se han desarrollado múltiples trabajos relacionados con la investigación de las emociones desde múltiples campos como puede ser la psicología, la medicina y las ciencias de la computación. Los trabajos relacionados con este último campo se engloban dentro de lo que se conoce como la Computación Afectiva, y entre sus objetivos se encuentran el desarrollo de sistemas capaces de de reconocer, interpretar, procesar y estimular las emociones de las personas \url{https://www.bbvaopenmind.com/tecnologia/mundo-digital/que-es-la-computacion-afectiva/} El objetivo detrás de esta interpretación de las emociones suele ser el poder adaptar el comportamiento de los programas para dar una respuesta adecuada a esas emociones.

Existen dos métodos principales a la hora de clasificar las emociones humanas:
\begin{itemize}
    \item Continua: Se emplea una dimensión que podría ser positiva-negativa o calmado-excitado.
    \item Categórica: Se emplean clases discretas que podrían ser feliz, triste, enfadado, etc. Y se emplea algún método de aprendizaje automático para llevar a cabo un modelo que permita clasificar entre las distintas categorías.
\end{itemize}

\subsection{Sentimiento y Análisis de Sentimiento}

Análisis de Sentimiento consiste en el uso de biometría, procesamiento de lenguaje natural, análisis de voz, imágenes o vídeos, etc. con el fin de extraer, identificar y cuantificar estados afectivos o información subjetiva 
\url{https://en.wikipedia.org/wiki/Sentiment_analysis}

Esta información para múltiples propósitos como puede ser el fin de mejorar el servicio, la atención al cliente o el marketing de productos

\subsection{Red Social}
RAE: red social
Servicio de la sociedad de la información que ofrece a los usuarios una plataforma de comunicación a través de internet para que estos generen un perfil con sus datos personales, facilitando la creación de comunidades con base en criterios comunes y permitiendo la comunicación de sus usuarios, de modo que pueden interactuar mediante mensajes, compartir información, imágenes o vídeos, permitiendo que estas publicaciones sean accesibles de forma inmediata por todos los usuarios de su grupo.


\tablaSmall{Herramientas y tecnologías utilizadas en cada parte del proyecto}{l c c c c}{herramientasportipodeuso}




{ \multicolumn{1}{l}{Herramientas} & App AngularJS & API REST & BD & Memoria \\}{ 
HTML5 & X & & &\\
CSS3 & X & & &\\
BOOTSTRAP & X & & &\\
JavaScript & X & & &\\
AngularJS & X & & &\\
Bower & X & & &\\
PHP & & X & &\\
Karma + Jasmine & X & & &\\
Slim framework & & X & &\\
Idiorm & & X & &\\
Composer & & X & &\\
JSON & X & X & &\\
PhpStorm & X & X & &\\
MySQL & & & X &\\
PhpMyAdmin & & & X &\\
Git + BitBucket & X & X & X & X\\
Mik\TeX{} & & & & X\\
\TeX{}Maker & & & & X\\
Astah & & & & X\\
Balsamiq Mockups & X & & &\\
VersionOne & X & X & X & X\\
} 
