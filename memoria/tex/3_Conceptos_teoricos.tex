\capitulo{3}{Conceptos teóricos}

En este apartado se presentan los distintos conceptos teóricos necesarios para la correcta comprensión del proyecto.

\section{Emoción y Sentimiento}

A pesar de que en el lenguaje popular en muchas ocasiones se usan ambas palabras de forma equivalente, no lo son. Las emociones y los sentimientos representan dos conceptos distintos aunque relacionados.

Por un lado, las emociones son un conjunto de reacciones neurofisiológicas que sufre un individuo al ser expuesto a algún objeto, lugar, persona, suceso o idea. Las emociones son transitorias, se dan de forma inconsciente y espontánea. En múltiples ocasiones las emociones son el origen de los cambios en la motivación y el comportamiento de los individuos. En ocasiones se dividen las emociones entre primarias o básicas y secundarias, siendo las 4 emociones primarias el miedo, la tristeza, el enfado y la alegría \cite{psicoemocionat}.

Por otro lado, los sentimientos son un estado afectivo que se genera a partir de una emoción \cite{psicoonline}. En el sentimiento interviene tanto la reacción neurofisiológica como un componente cognitivo. Esto significa que mientras que la emoción se daba de forma instintiva, en el sentimiento interviene un razonamiento que lo hace más duradero.

En resumen, las principales diferencias entre emoción y pensamiento radican en:
\begin{enumerate}
    \item Mientras que las emociones duran poco tiempo los sentimientos son duraderos.
    \item La emoción precede al sentimiento.
    \item Los sentimientos se producen instintivamente mientras que en los sentimientos interviene el pensamiento, el cual es afectado por la experiencia previa del individuo, características subjetivas, etc.
\end{enumerate}

\section{Análisis de Emociones y de Sentimientos}

Al igual que sucedía en el apartado anterior, los conceptos de análisis de emoción y de sentimiento a pesar de parecer similares no lo son.

Primero, el análisis de emociones busca descubrir la reacción que puede tener un individuo ante cierto estímulo. Como se comentó en el anteriormente, las emociones en muchas ocasiones son el motivo de las acciones de las personas, con lo que conocer si una persona se encuentra aburrida, contenta o enfadada puede ser de utilidad para dar una respuesta adecuada.

Por otro lado, el análisis de sentimiento no busca descubrir que emoción en particular es expresada por los usuarios, si no más bien conocer si la respuesta de un individuo es positiva o negativa, lo que se suele conocer como polaridad (en ocasiones también se incluye la neutralidad) \cite{analyticssteps}. Ejemplos de este tipo de análisis pueden ser el conocer el éxito de una campaña publicitaria, la recepción de una película, serie, noticia, etc.

A lo largo de los últimos años se han desarrollado múltiples trabajos relacionados con el análisis de emociones y de sentimiento desde distintos campos como puede ser la psicología, la medicina y las ciencias de la computación. Los trabajos relacionados con este último campo se engloban dentro de lo que se conoce como la \textit{Computación Afectiva}, y entre sus objetivos se encuentran el desarrollo de sistemas capaces de de reconocer, interpretar, procesar y estimular las emociones de las personas \cite{bbva_comp_afec}. El objetivo detrás de esta interpretación de las emociones suele ser el poder adaptar el comportamiento de los programas para dar una respuesta adecuada a esas emociones. Para ello se puede hacer uso de biometría, procesamiento de lenguaje natural, análisis de voz, imágenes, vídeos, etc. con el fin de extraer, identificar y cuantificar estados afectivos o información subjetiva. Esta información puede ser empleada para múltiples propósitos como puede ser mejorar un servicio, la atención al cliente o el marketing de productos.

\section{Red social}

Existen múltiples definiciones de red social. La RAE define a las redes sociales como un \guillemotleft servicio de la sociedad de la información que ofrece a los usuarios una plataforma de comunicación a través de internet\guillemotright. Este servicio permite la comunicación de sus usuarios mediante mensajes, audio, imágenes o vídeos, lo que facilita la creación de comunidades con gustos similares.

Por otro lado, el INTECO en su documento ``Estudio sobre la privacidad de los datos
personales y la seguridad de la información en las redes sociales online'' define a las redes sociales como unos \guillemotleft servicios prestados a través de Internet que permiten a los usuarios generar un perfil público, en el que plasmar datos personales e
información de uno mismo, disponiendo de herramientas que permiten interactuar
con el resto de usuarios afines o no al perfil publicado\guillemotright.

Sobre su utilidad, en \cite{ontsi_redes_sociales} el ONTSI afirma que las redes sociales sirven como una herramienta de \guillemotleft democratización de la información que transforma a las personas en receptores y en productores de contenidos\guillemotright.

Hay múltiples tipos de redes sociales de acuerdo a distintos criterios de clasificación:
\begin{itemize}
    \item Según el nivel de integración, podemos tener redes sociales \textit{horizontales} donde no existe una temática definida o \textit{verticales} donde los temas de conversación están dirigidos hacia un tema en específico con el fin de juntar a un grupo de usuarios que tengan interés en dicho tipo de contenido, como podrían ser los foros especializados.
    \item Según la finalidad, podríamos clasificar las redes sociales entre aquellas que son de \textit{ocio}, donde se busca el entretenimiento, ya sea hablando sobre algun tema como podría ser música, películas o videojuegos, o simplemente conversando de forma libre, y por otro lado podríamos tener redes \textit{profesionales} con el fín de promocionarse o la búsqueda de empleo.
    \item Según su apertura podemos tener redes sociales \textit{públicas} donde cualquiera con acceso a internet puede comentar ya sea con o sin necesidad de registrarse, o redes \textit{privadas} donde el grupo de usuarios que puede acceder es cerrado.
    \item Según el modo de funcionamiento, donde se pueden clasificar entre redes sociales de \textit{contenido} donde se busca compartir documentos, vídeos, imágenes... Las redes basadas en \textit{microblogging} donde el contenido compartido consiste normalmente en mensajes contenidos que facilitan el seguimiento por parte de los usuarios. Y por último aquellas basadas en \textit{perfiles} donde se busca compartir contenido personal o profesional mediante la creación de perfiles o fichas de usuario.
\end{itemize}

Las redes sociales se han popularizado mucho en los últimos años. De acuerdo con el artículo de wikipedia sobre las redes sociales, las más populares en la actualidad son las siguientes:

\begin{table}[h]
    \centering
    \begin{tabular}{|l|l|}
        \hline
        Nombre    & Número de usuarios (en millones) \\ \hline
        Facebook  & 2910              \\ \hline
        YouTube   & 2562              \\ \hline
        WhatsApp  & 2000              \\ \hline
        Instagram & 1478              \\ \hline
        WeChat    & 1263              \\ \hline
    \end{tabular}
    \caption{Ránking de redes sociales a Enero de 2022}
    \label{tab:socmed_ranking}
\end{table}

\subsection{Instagram}

Instagram es una red social estadounidense, en la actualidad propiedad del grupo Meta, lanzada en Octubre de 2010. Originalmente se lanzó como aplicación exclusiva de los móviles iPhone, aunque en la actualidad tiene aplicaciones para otras plataformas móviles como Android, así como versión web accesible desde cualquier dispositivo. El objetivo principal de esta red social es compartir imágenes o vídeo entre usuarios, ya sea de forma pública o entre seguidores autorizados, aunque en la actualidad Instagram ofrece más servicios como la mensajería directa entre usuarios, subir contenido de duración limitada con \textit{Instagram Stories} o el streaming de vídeo mediante IGTV. Dentro de la clasificación anteriormente expuesta, Instagram sería una red horizontal, de ocio, pública y basada en contenido.

Instagram permite el etiquetado de su contenido mediante el uso de \textit{hashtags}, con el fin de facilitar la búsqueda de contenido similar y el descubrimiento de cuentas de usuario que comparten publicaciones sobre dicho tema, permitiendo crear comunidades de personas con gustos afines.

\section{Web scraping}

El web scraping, o raspado web, es una técnica empleada para extraer datos de páginas web. Aunque esta acción puede ser llevada a cabo manualmente por un usuario, normalmente esta técnica se refiere a procesos automatizados mediante software como pueden ser \textit{bots} o \textit{crawlers}, los cuales hacen uso de protocolo HTTP para acceder a las páginas web simulando el comportamiento que haría una persona. Normalmente, este tipo de programas extraen, parsean y transforman la información mostrada por la página web (generalmente de forma no estructurada) para después almacenarla en hojas de cálculo o bases de datos. Existen múltiples técnicas para llevar a cabo web scrapping, aunque las más comunes consisten en llevar a cabo búsqueda de patrones o parsear los documentos HTML de las páginas web para buscar ciertas etiquetas.

\section{Visión por computador}

La visión por computador, también conocida como visión artificial, es el campo de las ciencias de la computación encargado analizar, comprender y responder a imágenes o vídeos \cite{insight_cv}. Para ello se suele hacer uso de técnicas de aprendizaje profundo para buscar patrones en la entrada, de modo que tras una fase de entrenamiento, el modelo generado sea capaz de reconocer, clasificar y reaccionar a lo que ``ve''. La visión artificial es un campo en crecimiento, con un potencial enorme puesto que puede ser aplicada en entornos donde se requiera una monitorización constante, ya que permite evitar evitar errores humanos derivados de distracciones o fatiga. Entre las aplicaciones más comunes en la actualidad y con más potencial en el futuro se encuentran el control de calidad en procesos de manufactura, la detección de intrusos, diagnosis médica, conducción autónoma, etc.

\section{Dashboard}

Un dashboard o cuadro de mando es un tipo de interfaz de usuario capaz de mostrar de forma integrada múltiples indicadores clave de rendimiento (KPIs) que son de interés para un proceso de negocio. El uso de cuadros de mando es muy común en la inteligencia de negocios para ayudar a visualizar datos complejos. Esto es debido a que los cuadros de mando permiten:

\begin{itemize}
    \item Mostrar múltiples visualizaciones simultáneamente.
    \item Usar diversos tipos de gráficos de distinto tipo.       
    \item Interaccionar con los datos, generalmente mediante filtros que permiten focalizar la representación en una serie de datos que puedan ser de interés del usuario.
\end{itemize}

A mayores, los cuadros de mandos pueden permitir resaltar o hacer anotaciones, crear alarmas o notificaciones si los datos salen fuera de un umbral o cota de referencia, o refrescar los datos presentados de forma automática según estos se van actualizando.

El objetivo principal de los cuadros de mandos es ayudar a sus usuarios a la gestión facilitando la comprensión de los datos. De acorde al tipo de usuario que va a emplear un cuadro de mando, éstos se podrían clasificar dentro de las siguientes categorías:

\begin{itemize}
    \item Paneles estratégicos: son usados por el personal directivo y se usan para monitorizar el grado de cumplimiento de la estrategia de una empresa en base a distintos KPIs. En este tipo de panel se trabaja a largo plazo, y generalmente son complejos de crear.
    \item Paneles operacionales: Son usados por empleados o mandos intermedios para monitorizar y gestionar tareas relacionadas con su trabajo diario. En este tipo de panel se trabaja en intervalos de tiempo más cortos, y son el tipo de panel más habitual.
    \item Paneles analíticos: son usados por analistas y directivos intermedios, se usan para ayudar a interpretar la información de grandes cantidades de datos, principalmente para comparar el rendimiento de datos históricos para poder predecir comportamientos o tendencias a corto y medio plazo.
\end{itemize}
