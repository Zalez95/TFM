\capitulo{2}{Objetivos del proyecto}

En este apartado se exponen los distintos objetivos que se pretenden satisfacer en el proyecto actual. Estos objetivos se han dividido en tres tipos: generales, técnicos y personales, los cuales se describen a continuación.

\section{Objetivos generales}

Los objetivos principales planteados para el proyecto son los siguientes:

\begin{itemize}
    \item Conocer el estado del arte, investigando artículos que ya hayan sido realizados en relación a la recomendación de publicaciones o lugares donde hacer turismo en base a contenido de redes sociales.
    \item Desarrollar el software necesario para descargar publicaciones de Instagram y su posterior almacenamiento.
    \item Usar algún servicio de reconocimiento de imágenes para llevar a cabo análisis de sentimiento sobre las publicaciones, así como la extracción de información relativa a las personas presentes en la imagen, como su edad o género.
    \item Implementar o usar algún medio para llevar a cabo la visualización y filtrado de los datos de forma sencilla.
\end{itemize}

\section{Objetivos técnicos}

En base a los objetivos anteriores se plantean los siguientes requerimientos técnicos:

\begin{itemize}
    \item Se busca el empleo de herramientas en la nube para el alojamiento y procesamiento de los datos.
    \item El desarrollo del proyecto no ha de suponer ningún coste, por ello se pretende usar software libre, herramientas gratuitas o herramientas que tengan un periodo de prueba lo suficientemente extenso como para poder desarrollar el proyecto sin complicaciones.
    \item Para la visualización de los datos se busca desarrollar un cuadro de mandos con alguna herramienta o servicio ya existente.
\end{itemize}

\section{Objetivos personales}

Entre los objetivos personales que se buscan satisfacer con este proyecto se encuentran los siguientes:

\begin{itemize}
    \item Familiarizarse con el uso de herramientas y servicios populares de computación en la nube.
    \item Ampliar mis conocimientos del lenguaje de programación Python así como aprender a usar nuevas librerías e interfaces de programación.
    \item Poner en práctica mis conocimientos sobre cuadros de mando, profundizando en el uso de herramientas utilizadas a lo largo del máster o aprendiendo a usar otras nuevas.
    \item Mejorar mis conocimientos sobre almacenamiento y procesado de grandes cantidades de datos.
\end{itemize}
